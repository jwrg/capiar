\documentclass[10pt]{article}
%
% Base header file
%
% For use with basically all templates
%
% Provides basic functionality that is common to all
% LaTeX files authored using this template hierarchy.
%
% Anything which is found to not be used regularly or
% is specific to one use, it should be pruned.
%
% NB that geometry is absent here so that it can be fed
% parameters for different use cases.
%

% Fonts and symbols
\usepackage{
  amsmath,			% math operators
  amssymb,			% math symbols
  textcomp,			% Copyright and Registered symbols
  pifont,				% Includes the pretty cirled numbers
}

% Layout
\usepackage{
  multicol,			% allow multi columns
  parskip,			% no indenting on paras with a line between paras
  fancyhdr,			% fancy headers and footers
  setspace,			% double- and one-half-spacing
}

% Tables
\usepackage{
  tabu,         % better than tabularx
  booktabs,     % improves table appearance
}

%
% Dictionary and linguistic-related header file
%
% Provides packages and commands for authoring dictionaries and
% other such language-related resource files
%
% Includes:
%
% res/style/font.tex (shiny fonts)
% res/style/colour.tex (neat colours)
% res/macro/langtab.tex (for language-related tables)
% res/macro/langdict.tex (for dictionary authoring)
%

% Layout
\usepackage{
  environ,      % enhanced environment declaration
  enumitem,     % fancy enumerators
  geometry,     % adjust margins, portrait layout
}

\geometry{
  letterpaper,		% paper size
  margin=1.25in,	% specified independently with hmargin vmargin
}	

% Overrides
\renewcommand{\arraystretch}{1.5}
\setlength{\tabcolsep}{12pt}

% Fonts
%
% These are the fonts I know, I know
%

\usepackage[sfdefault]{raleway}
\usepackage{inconsolata}
\usepackage{librebaskerville}
\usepackage[T1]{fontenc}


% Colours
%
% Colour resource file
%
% Declare some colours and redeclare some stuff to make it pretty
%
\usepackage[dvipsnames]{
  xcolor,       % Dope colours
}

\usepackage{
  colortbl,     % Colours in tables
}

\renewcommand{\bullet}{\textcolor{pink}{\blacktriangleright}}

\definecolor{purple}{cmyk}{0,0.71,0.58,0.52}
\definecolor{brown}{cmyk}{0,0.02,0.38,0.58}
\definecolor{pink}{cmyk}{0,0.61,0.5,0.28}
\definecolor{blue}{cmyk}{0.71,0.21,0,0.65}
\definecolor{grey}{cmyk}{0.19,0.6,0,0.80}
\definecolor{lgrey}{cmyk}{0.19,0.6,0,0.20}



% Includes 
%
% Language- and linguistic-related table file
%
% Provides tables for language assignments and presentations
% 
% Included in langhead.tex
%
% verbchart{<verb>}{<tense>}{<voice>}{<mood>}
%   -> This environment will set up a verb chart for conjugating
%      a noun; all args mandatory
%
% nounchart{<noun>}{<gender>}{<declension>}
%   -> This environment sets up a noun chart for declining
%      a noun (or an adjective, I suppose); all args 
%      mandatory (NEEDS CHANGING)
%
% nounphrasechart{<noun>}{<gender>}{<noun declension>}{<adjective>}{<adj declension>}
%   -> This environment sets up a chart for declining a noun
%      and an adjective together as a phrase; all args
%      mandatory (ALSO NEEDS CHANGING)

% Environments
\NewEnviron{refchart}{
  \begin{center}
    \renewcommand{\arraystretch}{2}
    \taburowcolors 2{black!10 .. white}
    \begin{tabu}to \linewidth {lX[r]}
      \BODY
    \end{tabu}
  \end{center}
}

\NewEnviron{verbchart}[4]{
  \begin{center}
    \setlength{\tabcolsep}{4pt}
    \begin{tabu}to \textwidth {X[l]X[0.7,c]X[0.8,c]X[0.6,r]}
    \textit{{#1}} & \textbf{Tense:} {#2} & \textbf{Mood:} {#4} & \textbf{Voice:} {#3} \\ 
  \end{tabu}
  \end{center}

  \vspace{-1em}
  \begin{center}
  \textsf{
    \begin{tabu}to \textwidth {|X[c]|X[2.5,l]|X[2.5,l]|}
      \hline
      & \textbf{Singular} & \textbf{Plural} \\
      \hline
      \BODY
    \end{tabu}
    \vspace{3em}
  }
  \end{center}
  }

  \NewEnviron{nounchart}[3]{
    \begin{center}
      \setlength{\tabcolsep}{4pt}
      \begin{tabu} to \textwidth {X[1.1,l]X[r]}
        \textit{{#1}},  {#2}. & {#3} declension \\ 
      \end{tabu}
    \end{center}

    \vspace{-1em}
    \begin{center}
    \textsf{
      \begin{tabu}to \textwidth {|X[c]|X[1.2,l]|X[1.2,l]|}
        \hline
        & \textbf{Singular} & \textbf{Plural} \\
        \hline
        \BODY
      \end{tabu} 
      \vspace{3em}
    }
    \end{center}
  }

  \NewEnviron{nounphrase}[5]{
    \begin{center}
      \setlength{\tabcolsep}{4pt}
      \begin{tabu} to \textwidth {X[1.2,l]X[c]X[1.2,c]X[r]}
        \textit{{#1}},  {#2}. & {#3} declension & \textit{{#4}} & {#5} declension \\ 
      \end{tabu}
    \end{center}

        \vspace{-1em}
    \begin{center}
      \textsf{
        \begin{tabu} to \textwidth {|X[c]|X[1.2,l]|X[1.2,l]|}
          \hline
          \hfill & \textbf{Singular} & \textbf{Plural} \\
          \hline
          \BODY
        \end{tabu}
        \vspace{3em}
      }
    \end{center}
  }

%
% Language- and linguistic-related dictionary file
%
% Provides some commands useful for authoring dictionary
% documents.
%
% titlepage{<title>}{<subtitle>}{<author>}
%   ->  This command provides a title page.  For use at the
%       front of the document.  Subject to refinement.
%
% dictentry{<lhs>}{<rhs>}
%   ->  This command simplifies the generation of dictionary-
%       style entries without the use of tables.
%

\makeatletter
\def\removespaces#1{\zap@space#1 \@empty}
\makeatother

\newcommand{\+}{\discretionary{}{}{}}

\newcommand{\firstpage}[3]{
  \begin{center}
  \uppercase{\fontsize{76pt}{78pt}\selectfont\removespaces{#1}}
  \fontsize{10pt}{12pt}\selectfont

    \vspace{6em}
    { \normalsize {#2} }

    \vspace{6em}
    \uppercase{\bfseries\Large {#3} }
  \end{center}
}

\newcommand{\dictentry}[2]{
  \makebox[\linewidth]{
  \noindent\begin{minipage}[c][][c]{0.66\linewidth}
    \raggedright
    \textbf{ #1 }
  \end{minipage}
  \dotfill
  \begin{minipage}[c][][c]{0.33\hsize}
    \raggedleft
    { #2 }
  \end{minipage}
  }

\vspace{1ex}
}



\begin{document}
\pagenumbering{gobble}		% turn off page numbering

\titletable{Latin 101}{Dr. Molinarius}{Assignment 14}{}{Iacomus Guilliamus}{Robertus Gallovidius}

\begin{enumerate}
  \item Decline the following noun-adjective combinations:

    \begin{enumerate}
      \item daring poet

        \begin{nounphrase}{po\=eta, po\=etae}{m}{1st}{audax, audacis}{3rd}
          Nominative  & po\=eta aud\={a}x       & po\=etae aud\={a}c\=es \\\hline
          Genitive    & po\=etae aud\={a}cis    & po\=et\=arum aud\={a}cium \\\hline
          Dative      & po\=etae aud\={a}c\=i   & po\=et\=is aud\={a}cibus \\\hline
          Accusative  & po\=etam aud\={a}cem    & po\=et\=as aud\={a}c\=es \\\hline
          Ablative    & po\=et\=a aud\={a}c\=i  & po\=et\=is aud\={a}cibus \\\hline
        \end{nounphrase}

      \item strong man

        \begin{nounphrase}{vir, vir\=i}{m}{2nd}{fortis, fortis}{3rd}
          Nominative  & vir fortis      & vir\=i fort\=es \\\hline
          Genitive    & vir\=i fortis   & vir\=orum fortium \\\hline
          Dative      & vir\=o fort\=i  & vir\=is fortibus \\\hline
          Accusative  & virum fortem    & vir\=os fort\=es \\\hline
          Ablative    & vir\=o fort\=i  & vir\=is fortibus \\\hline
        \end{nounphrase}

      \item sharp wound

        \begin{nounphrase}{vulnus, vulneris}{n}{3rd}{\=acre, \=acris}{3rd}
          Nominative  & vulnus \=acre       & vulnera \=acria \\\hline
          Genitive    & vulneris \=acris    & vulnerum \=acrium \\\hline
          Dative      & vulner\=i \=acr\=i  & vulneribus \=acribus \\\hline
          Accusative  & vulnus \=acre       & vulnera \=acria \\\hline
          Ablative    & vulnere \=acr\=i    & vulneribus \=acribus \\\hline
        \end{nounphrase}
    \end{enumerate}

  \item Translate the following sentences into Latin:

    \begin{tabu}to \linewidth {XX}
      of the easy language & linguae facilis \\
      to/for the fierce legions & legionibus \=acribus \\
      under the beautiful sky & sub cael\=o pulchr\=o \\
      in great grief & in dol\=ore magn\=o \\
      by/with/from strong bodies & corporibus fortibus \\
      of the bold girls & puell\=arum aud\=acium \\
      short months (as subject) & mens\=es br\=ev\=es \\
      by/with/from a strong mind & mente fort\=i \\
      easy wars (direct object) & bella facilia \\
      into the enemy's fierce city & in urbem \=acrem hostis \\
    \end{tabu}

  \item Translate the following sentences into English:

    \begin{tabu}to \linewidth {XX}
      Vulnus m\=ilit\=i dol\=orem facit. & The wound causes pain to the soldier. \\
      Virt\=utem f\=emin\=arum fortium laud\=emus. & Let us praise the courage of the strong women. \\
      \=E dol\=oribus \=acribus disc\=i potest v\=erit\=as. & From sharp pains can the truth be learned. \\
      \=Acr\=es ment\=es habent linguae Lat\=inae discipul\=i et discipulae. & The male and female students of the Latin language have sharp minds. \\
      L\=uce s\=olis fl\=or\=es pulchr\=i gignuntur. & The beautiful flowers are beget by the light of the sun. \\
      \=Acr\=i in bell\=o vir\=i fort\=es bon\=ique caeduntur. & The strong and good men are slain in the fierce war. \\
      In locum miserum congregantur servae domin\=i \=acris. & The female slaves of the fierce lord are gathered in the miserable location. \\
      Fort\=es este, puer\=i et puellae! & Be strong, boys and girls! \\
      Deum aeternum, n\=on v\=itam brevem, col\=amus. & Let us worship everlasting god, not short life. \\
      V\=erit\=atem vid\=ere facile n\=on est. & Truth is not easy to see. \\
    \end{tabu}
\end{enumerate}
\end{document}
