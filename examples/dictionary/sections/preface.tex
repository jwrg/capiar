Preface to the First Edition

\begin{itemize}
  \item The modern convention of using macrons (\=a, \=o)
    instead of apices (\'a, \'o).
  \item The intention of making this document a quick reference
    for common Latin forms, and not an exhaustive compendium
  \item Also this dictionary is not meant to be a guide for the
    early beginner, but for one who already has a rudimentary
    grasp of the language.
  \item The core of the text comes from Wheelock and Molinarius
  \item Basically once I encounter something twice in my readings,
    it gets added until I feel like the text is a somewhat
    complete reference
  \item Elaborating on the previous item, the first time a word
    is added to the reference, it is added but as a comment.  When
    it gets encountered again, the comment can simply be removed
    as I think it's likely that I'll forget how many times I come
    across a word without having some way to otherwise keep track.
  \item The intention is for this to be useful for me, and if others
    find it useful as well, then I'll feel like I've accomplished
    something.
  \item Elect to eschew the lexicographer's fetish for abbreviation
    and brevity in favour of legibility and intelligibility.  It's
    not as though I am destitute on space in this document.
  \item Instead of abbreviations, leverage typography (e.g., the
    \textbackslash dictontry command which instead of displaying the terms
    side-by-side, shows them as alternating lines of raggedright
    and raggedleft, in the hopes that neither side of the entry
    takes up more than a single line.  (Did this allow for more
    columns?)
  \item Mulling over the idea of whether to sort verbs
    lexicographically by first-person active indicative (as
    is tradition) or to sort by infinitive, which makes
    more sense in my dumb brain for some reason (think
    \textit{d\=o, d\=are, ded\=i, datum}, all principal parts
    having different vowels after the initial \textit{d}).
\end{itemize}
