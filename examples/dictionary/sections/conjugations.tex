\section{Conjugations}
Latin verbs can be categorized into four \textit{conjugations} which are
quickly discerned by their differing infinitive forms (i.e., look at the
second principle part to determine the theme vowel).  This section will
enumerate the regular endings for each conjugations, and list some more
common deviations.

Before jumping into the individual conjugations, presented here are the
personal endings common to all conjugations.

\begin{verbchart}{Personal endings}{present}{active}{indicative}
  1 & -\textbf{o}, -\textbf{m}  & -\textbf{mus} \\\hline
  2 & -\textbf{s}      & -\textbf{tis} \\\hline
  3 & -\textbf{t}      & -\textbf{nt} \\\hline
\end{verbchart}

\begin{verbchart}{Personal endings}{present}{passive}{indicative}
  1 & -\textbf{r}   & -\textbf{mur} \\\hline
  2 & -\textbf{ris} & -\textbf{min\=i} \\\hline
  3 & -\textbf{tur} & -\textbf{ntur} \\\hline
\end{verbchart}

\newpage
\subsection{The First Conjugation (Long-\=a verbs)}
First conjugation verbs are characterized by the theme vowel, \=a.  Unlike
all other conjugations, the subjunctive is formed with a long \=e.

\begin{verbchart}{d\=o, d\=are, ded\=i, datum}{present}{active}{indicative}
  1 & d\textbf{\=o}    & d\=a\textbf{mus} \\\hline
  2 & d\=a\textbf{s}   & d\=a\textbf{tis} \\\hline
  3 & da\textbf{t}     & da\textbf{nt} \\\hline
\end{verbchart}

\begin{verbchart}{coni\=ur\=o, coni\=ur\=are, coni\=ur\=avi, coni\=ur\=atum}{present}{passive}{indicative}
  1 & coni\=uro\textbf{r}     & coni\=ur\=a\textbf{mur} \\\hline
  2 & coni\=ur\=a\textbf{ris} & coni\=ur\=a\textbf{min\=i} \\\hline
  3 & coni\=ura\textbf{tur}   & coni\=ura\textbf{ntur} \\\hline
\end{verbchart}

\begin{verbchart}{am\=o, am\=are, am\=avi, am\=atum}{present}{active}{subjunctive}
  1 & am\textbf{em}    & am\textbf{\=emus} \\\hline
  2 & am\textbf{\=es}  & am\textbf{\=etis} \\\hline
  3 & am\textbf{\=et}  & am\textbf{ent} \\\hline
\end{verbchart}

\begin{verbchart}{laud\=o, laud\=are, laud\=avi, laud\=atum}{present}{passive}{subjunctive}
  1 & laud\textbf{er}     & laud\textbf{\=emur} \\\hline
  2 & laud\textbf{\=eris} & laud\textbf{\=emin\=i} \\\hline
  3 & laud\textbf{\=etur} & laud\textbf{entur} \\\hline
\end{verbchart}

\begin{verbchart}{d\=o, d\=are, ded\=i, datum}{future}{active}{indicative}
  1 & d\=a\textbf{b\=o}    & d\=a\textbf{bimus} \\\hline
  2 & d\=a\textbf{bis}     & d\=a\textbf{bitis} \\\hline
  3 & d\=a\textbf{bit}     & d\=a\textbf{bunt} \\\hline
\end{verbchart}

\begin{verbchart}{coni\=ur\=o, coni\=ur\=are, coni\=ur\=avi, coni\=ur\=atum}{future}{passive}{indicative}
  1 & coni\=ur\=a\textbf{bor}     & coni\=ur\=a\textbf{bimur} \\\hline
  2 & coni\=ur\=a\textbf{beris}   & coni\=ur\=a\textbf{bimin\=i} \\\hline
  3 & coni\=ur\=a\textbf{bitur}   & coni\=ur\=a\textbf{buntur} \\\hline
\end{verbchart}

\begin{verbchart}{d\=o, d\=are, ded\=i, datum}{imperfect}{active}{indicative}
  1 & d\=a\textbf{bam}   & d\=a\textbf{b\=amus} \\\hline
  2 & d\=a\textbf{b\=as} & d\=a\textbf{b\=atis} \\\hline
  3 & d\=a\textbf{bat}   & d\=a\textbf{bant} \\\hline
\end{verbchart}

\begin{verbchart}{coni\=ur\=o, coni\=ur\=are, coni\=ur\=avi, coni\=ur\=atum}{imperfect}{passive}{indicative}
  1 & coni\=ur\=a\textbf{bar}     & coni\=ur\=a\textbf{b\=amur} \\\hline
  2 & coni\=ur\=a\textbf{b\=aris} & coni\=ur\=a\textbf{b\=amin\=i} \\\hline
  3 & coni\=ur\=a\textbf{b\=atur} & coni\=ur\=a\textbf{b\=antur} \\\hline
\end{verbchart}

\begin{verbchart}{am\=o, am\=are, am\=avi, am\=atum}{imperfect}{active}{subjunctive}
  1 & am\=ar\textbf{em}    & am\=ar\textbf{\=emus} \\\hline
  2 & am\=ar\textbf{\=es}  & am\=ar\textbf{\=etis} \\\hline
  3 & am\=ar\textbf{\=et}  & am\=ar\textbf{ent} \\\hline
\end{verbchart}

\subsection{The Second Conjugation (Long-\=e verbs)}
Second conjugations verbs are characterized by the theme vowel, \=e.  Easily
recognizable second conjugation verbs include \textit{vide\=o, vid\=ere,
v\=id\=i, v\=isum}, and \textit{habe\=o, hab\=ere, habu\=i, habitum}.

\begin{verbchart}{vide\=o, vid\=ere, v\={i}d\=i, v\=isum}{present}{active}{indicative}
  1 & vide\textbf{\=o}   & vid\=e\textbf{mus} \\\hline
  2 & vid\=e\textbf{s}   & vid\=e\textbf{tis} \\\hline
  3 & vide\textbf{t}     & vide\textbf{nt} \\\hline
\end{verbchart}

\begin{verbchart}{fove\=o, fov\=ere, fov\=i, f\=otum}{present}{passive}{indicative}
  1 & foveo\textbf{r}    & fov\=e\textbf{mur} \\\hline
  2 & fov\=e\textbf{ris} & fov\=e\textbf{min\=i} \\\hline
  3 & fove\textbf{tur}   & fove\textbf{ntur} \\\hline
\end{verbchart}

Unlike the first and third conjugations, and like the fourth conjugation,
the subjunctive for second conjugation verbs is formed with a long \=a 
\textit{after} a shortened theme vowel, in this case short-e.

\begin{verbchart}{habe\=o, hab\=ere, habu\=i, habitum}{present}{active}{subjunctive}
  1 & habe\textbf{am}    & habe\textbf{\=amus} \\\hline
  2 & habe\textbf{\=as}  & habe\textbf{\=atis} \\\hline
  3 & habe\textbf{\=at}  & habe\textbf{ant} \\\hline
\end{verbchart}

\begin{verbchart}{fove\=o, fov\=ere, fov\=i, f\=otum}{present}{passive}{subjunctive}
  1 & fove\textbf{ar}      & fove\textbf{\=amur} \\\hline
  2 & fove\textbf{\=aris}  & fove\textbf{\=amin\=i} \\\hline
  3 & fove\textbf{\=atur}  & fove\textbf{antur} \\\hline
\end{verbchart}

\begin{verbchart}{vide\=o, vid\=ere, v\={i}d\=i, v\=isum}{future}{active}{indicative}
  1 & vid\=e\textbf{b\=o}   & vid\=e\textbf{bimus} \\\hline
  2 & vid\=e\textbf{bis}    & vid\=e\textbf{bitis} \\\hline
  3 & vid\=e\textbf{bit}    & vide\textbf{bunt} \\\hline
\end{verbchart}

\begin{verbchart}{fove\=o, fov\=ere, fov\=i, f\=otum}{future}{passive}{indicative}
  1 & fov\=e\textbf{bor}    & fov\=e\textbf{bimur} \\\hline
  2 & fov\=e\textbf{beris}  & fov\=e\textbf{bimin\=i} \\\hline
  3 & fov\=e\textbf{bitur}  & fov\=e\textbf{buntur} \\\hline
\end{verbchart}

\begin{verbchart}{vide\=o, vid\=ere, v\={i}d\=i, v\=isum}{imperfect}{active}{indicative}
  1 & vid\=e\textbf{bam}    & vid\=e\textbf{b\=amus} \\\hline
  2 & vid\=e\textbf{b\=as}  & vid\=e\textbf{b\=atis} \\\hline
  3 & vid\=e\textbf{bat}    & vid\=e\textbf{bant} \\\hline
\end{verbchart}

\begin{verbchart}{fove\=o, fov\=ere, fov\=i, f\=otum}{imperfect}{passive}{indicative}
  1 & fov\=e\textbf{bar}      & fov\=e\textbf{b\=amur} \\\hline
  2 & fov\=e\textbf{b\=aris}  & fov\=e\textbf{b\=amin\=i} \\\hline
  3 & fov\=e\textbf{b\=atur}  & fov\=e\textbf{bantur} \\\hline
\end{verbchart}

\begin{verbchart}{habe\=o, hab\=ere, habu\=i, habitum}{imperfect}{active}{subjunctive}
  1 & hab\=e\textbf{rem}    & hab\=e\textbf{r\=emus} \\\hline
  2 & hab\=e\textbf{r\=es}  & hab\=e\textbf{r\=etis} \\\hline
  3 & hab\=e\textbf{ret}    & hab\=e\textbf{rent} \\\hline
\end{verbchart}

\subsection{The Third Conjugation (Short-e verbs)}
The third conjugation of Latin verbs is arguably the least regular, and is
characterized by the theme vowel short-e.  Notably, this short-e tends to
manifest itself as a short-i, as in the present active indicative.

\begin{verbchart}{d\=ic\=o, d\=icere, d\=ix\=i, dictum}{present}{active}{indicative}
  1 & d\=ic\textbf{\=o}    & d\=ici\textbf{mus} \\\hline
  2 & d\=ici\textbf{s}     & d\=ici\textbf{tis} \\\hline
  3 & d\=ici\textbf{t}     & d\=icu\textbf{nt} \\\hline
\end{verbchart}

\begin{verbchart}{ag\=o, agere, \=eg\=i, \=actum}{present}{passive}{indicative}
  1 & ago\textbf{r}     & agi\textbf{mur} \\\hline
  2 & age\textbf{ris}   & agi\textbf{min\=i} \\\hline
  3 & agi\textbf{tur}   & agu\textbf{ntur} \\\hline
\end{verbchart}

\begin{verbchart}{p\=on\=o, p\=onere, posu\=i, positum}{present}{active}{subjunctive}
  1 & p\=on\textbf{am}    & p\=on\textbf{\=amus} \\\hline
  2 & p\=on\textbf{\=as}  & p\=on\textbf{\=atis} \\\hline
  3 & p\=on\textbf{\=at}  & p\=on\textbf{ant} \\\hline
\end{verbchart}

\begin{verbchart}{ed\=o, edere, \=ed\=i, esum}{present}{passive}{subjunctive}
  1 & ed\textbf{ar}     & ed\textbf{\=amur} \\\hline
  2 & ed\textbf{\=aris} & ed\=a\textbf{min\=i} \\\hline
  3 & ed\textbf{\=atur} & ed\textbf{antur} \\\hline
\end{verbchart}

\begin{verbchart}{p\=on\=o, p\=onere, posu\=i, positum}{future}{active}{indicative}
  1 & pon\textbf{am}    & pon\textbf{\=emus} \\\hline
  2 & pon\textbf{\=es}  & pon\=a\textbf{\=etis} \\\hline
  3 & pon\textbf{et}    & pon\textbf{ent} \\\hline
\end{verbchart}

\begin{verbchart}{ed\=o, edere, \=ed\=i, esum}{present}{future}{indicative}
  1 & ed\textbf{ar}     & ed\textbf{\=emur} \\\hline
  2 & ed\textbf{\=eris} & ed\=a\textbf{\=emin\=i} \\\hline
  3 & ed\textbf{\=etur} & ed\textbf{entur} \\\hline
\end{verbchart}

\begin{verbchart}{d\=ic\=o, d\=icere, d\=ix\=i, dictum}{imperfect}{active}{indicative}
  1 & d\=ic\=e\textbf{bam}    & d\=ic\=e\textbf{b\=amus} \\\hline
  2 & d\=ic\=e\textbf{b\=as}  & d\=ic\=e\textbf{b\=atis} \\\hline
  3 & d\=ic\=e\textbf{bat}    & d\=ic\=e\textbf{bant} \\\hline
\end{verbchart}

\begin{verbchart}{ag\=o, agere, \=eg\=i, \=actum}{imperfect}{passive}{indicative}
  1 & ag\=e\textbf{bar}     & ag\=e\textbf{b\=amur} \\\hline
  2 & ag\=e\textbf{b\=aris} & ag\=e\textbf{b\=amin\=i} \\\hline
  3 & ag\=e\textbf{b\=atur} & ag\=e\textbf{bantur} \\\hline
\end{verbchart}

\begin{verbchart}{p\=on\=o, p\=onere, posu\=i, positum}{imperfect}{active}{subjunctive}
  1 & p\=one\textbf{rem}    & p\=one\textbf{r\=emus} \\\hline
  2 & p\=one\textbf{r\=es}  & p\=one\textbf{r\=etis} \\\hline
  3 & p\=one\textbf{ret}    & p\=one\textbf{rent} \\\hline
\end{verbchart}

\subsubsection{The Third\textit{-io}}
A subset of the third conjugation, the third-io can
be thought of as part third- and part fourth-conjugation.
It retains the theme vowel short-e but places an i in
some forms where it is not expected for a third conjugation
verb.

\begin{verbchart}{capi\=o, capere, c\=ep\=i, captum}{present}{active}{indicative}
  1 & capi\textbf{\=o}  & capi\textbf{mus} \\\hline
  2 & capi\textbf{s}    & capi\textbf{tis} \\\hline
  3 & capi\textbf{t}    & capiu\textbf{nt} \\\hline
\end{verbchart}

\begin{verbchart}{capi\=o, capere, c\=ep\=i, captum}{present}{passive}{indicative}
  1 & capio\textbf{r}   & capi\textbf{mur} \\\hline
  2 & capie\textbf{ris} & capi\textbf{min\=i} \\\hline
  3 & capi\textbf{tur}  & capiu\textbf{ntur} \\\hline
\end{verbchart}

\begin{verbchart}{capi\=o, capere, c\=ep\=i, captum}{present}{active}{subjunctive}
  1 & capi\textbf{am}   & capi\textbf{\=amus} \\\hline
  2 & capi\textbf{\=as} & capi\textbf{\=atis} \\\hline
  3 & capi\textbf{at}   & capi\textbf{ant} \\\hline
\end{verbchart}

\begin{verbchart}{capi\=o, capere, c\=ep\=i, captum}{present}{passive}{subjunctive}
  1 & capi\textbf{ar}     & capi\textbf{\=amur} \\\hline
  2 & capi\textbf{\=aris} & capi\textbf{\=amin\=i} \\\hline
  3 & capi\textbf{\=atur} & capi\textbf{antur} \\\hline
\end{verbchart}

\begin{verbchart}{capi\=o, capere, c\=ep\=i, captum}{future}{active}{indicative}
  1 & capi\textbf{am}     & capi\textbf{\=emus} \\\hline
  2 & capi\textbf{\=es}   & capi\textbf{\=etis} \\\hline
  3 & capi\textbf{et}     & capi\textbf{ent} \\\hline
\end{verbchart}

\begin{verbchart}{capi\=o, capere, c\=ep\=i, captum}{future}{passive}{indicative}
  1 & capi\textbf{ar}     & capi\textbf{\=emur} \\\hline
  2 & capi\textbf{\=eris} & capi\textbf{\=emin\=i} \\\hline
  3 & capi\textbf{\=etur} & capi\textbf{entur} \\\hline
\end{verbchart}

\begin{verbchart}{capi\=o, capere, c\=ep\=i, captum}{imperfect}{active}{indicative}
  1 & capi\=e\textbf{bam}   & capi\=e\textbf{b\=amus} \\\hline
  2 & capi\=e\textbf{b\=as} & capi\=e\textbf{b\=atis} \\\hline
  3 & capi\=e\textbf{bat}   & capi\=e\textbf{bant} \\\hline
\end{verbchart}

\begin{verbchart}{capi\=o, capere, c\=ep\=i, captum}{imperfect}{passive}{indicative}
  1 & capi\=e\textbf{bar}     & capi\=e\textbf{b\=amur} \\\hline
  2 & capi\=e\textbf{b\=aris} & capi\=e\textbf{b\=amin\=i} \\\hline
  3 & capi\=e\textbf{b\=atur} & capi\=e\textbf{bantur} \\\hline
\end{verbchart}

\begin{verbchart}{capi\=o, capere, c\=ep\=i, captum}{imperfect}{passive}{subjunctive}
  1 & cape\textbf{rem}    & cape\textbf{r\=emus} \\\hline
  2 & cape\textbf{r\=es}  & cape\=a\textbf{r\=etis} \\\hline
  3 & cape\textbf{ret}    & cape\textbf{rent} \\\hline
\end{verbchart}

\subsection{The Fourth Conjugation (Long-\=i verbs)}
The fourth conjugation is characterized by the theme vowel
long-\=i.

\begin{verbchart}{audi\=o, aud\=ire, aud\=iv\=i, auditum}{present}{active}{indicative}
  1 & audi\textbf{\=o}   & aud\=i\textbf{mus} \\\hline
  2 & aud\=i\textbf{s}   & aud\=i\textbf{tis} \\\hline
  3 & aud\=i\textbf{t}   & audiu\textbf{nt} \\\hline
\end{verbchart}

\begin{verbchart}{senti\=o, sent\=ire, s\=ens\=i, s\=ensum}{present}{passive}{indicative}
  1 & sentio\textbf{r}      & sent\=i\textbf{mur} \\\hline
  2 & sent\=i\textbf{ris}   & sent\=i\textbf{min\=i} \\\hline
  3 & sent\=i\textbf{tur}   & sentiu\textbf{ntur} \\\hline
\end{verbchart}

\begin{verbchart}{veni\=o, ven\=ire, v\=en\=i, ventum}{present}{active}{subjunctive}
  1 & veni\textbf{am}    & veni\textbf{\=amus} \\\hline
  2 & veni\textbf{\=as}  & veni\textbf{\=atis} \\\hline
  3 & veni\textbf{\=at}  & veni\textbf{ant} \\\hline
\end{verbchart}

\begin{verbchart}{senti\=o, sent\=ire, s\=ens\=i, s\=ensum}{present}{passive}{subjunctive}
  1 & senti\textbf{ar}       & senti\textbf{\=amur} \\\hline
  2 & senti\textbf{\=aris}   & senti\textbf{\=amin\=i} \\\hline
  3 & senti\textbf{\=atur}   & senti\textbf{antur} \\\hline
\end{verbchart}

\begin{verbchart}{audi\=o, aud\=ire, aud\=iv\=i, auditum}{future}{active}{indicative}
  1 & audi\textbf{am}     & audi\textbf{\=emus} \\\hline
  2 & audi\textbf{\=es} & audi\textbf{\=etis} \\\hline
  3 & audi\textbf{et}   & audi\textbf{ent} \\\hline
\end{verbchart}

\begin{verbchart}{senti\=o, sent\=ire, s\=ens\=i, s\=ensum}{future}{passive}{indicative}
  1 & senti\textbf{ar}      & senti\textbf{\=emur} \\\hline
  2 & senti\textbf{\=eris}  & senti\textbf{\=emin\=i} \\\hline
  3 & senti\textbf{\=etur}  & senti\textbf{entur} \\\hline
\end{verbchart}

\begin{verbchart}{audi\=o, aud\=ire, aud\=iv\=i, auditum}{imperfect}{active}{indicative}
  1 & audi\=e\textbf{bam}   & audi\=e\textbf{b\=amus} \\\hline
  2 & audi\=e\textbf{b\=as} & audi\=e\textbf{b\=atis} \\\hline
  3 & audi\=e\textbf{bat}   & audi\=e\textbf{bant} \\\hline
\end{verbchart}

\begin{verbchart}{senti\=o, sent\=ire, s\=ens\=i, s\=ensum}{imperfect}{passive}{indicative}
  1 & senti\=e\textbf{bar}      & senti\=e\textbf{b\=amur} \\\hline
  2 & senti\=e\textbf{b\=aris}  & senti\=e\textbf{b\=amin\=i} \\\hline
  3 & sentI\=e\textbf{b\=atur}  & senti\=e\textbf{bantur} \\\hline
\end{verbchart}

\begin{verbchart}{veni\=o, ven\=ire, v\=en\=i, ventum}{present}{active}{subjunctive}
  1 & veni\textbf{rem}    & veni\textbf{r\=emus} \\\hline
  2 & veni\textbf{r\=es}  & veni\textbf{r\=etis} \\\hline
  3 & veni\textbf{ret}    & veni\textbf{rent} \\\hline
\end{verbchart}
