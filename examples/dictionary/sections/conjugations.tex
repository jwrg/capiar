\section{Conjugations}
Latin verbs can be categorized into four \textit{conjugations} which are
quickly discerned by their differing infinitive forms (i.e., look at the
second principle part to determine the theme vowel).  This section will
enumerate the regular endings for each conjugations, and list some more
common deviations.

Before jumping into the individual conjugations, presented here are the
personal endings common to all conjugations.

\begin{verbchart}{Personal endings}{present}{active}{indicative}
  1 & -o, -m  & -mus \\\hline
  2 & -s      & -tis \\\hline
  3 & -t      & -nt \\\hline
\end{verbchart}

\begin{verbchart}{Personal endings}{present}{passive}{indicative}
  1 & -r   & -mur \\\hline
  2 & -ris & -min\=i \\\hline
  3 & -tur & -ntur \\\hline
\end{verbchart}

\newpage
\subsection{The First Conjugation (Long-\=a verbs)}
First conjugation verbs are characterized by the theme vowel, \=a.  Unlike
all other conjugations, the subjunctive is formed with a long \=e.

\begin{verbchart}{d\=o, d\=are, ded\=i, datum}{present}{active}{indicative}
  1 & d\=o    & d\=amus \\\hline
  2 & d\=as   & d\=atis \\\hline
  3 & dat     & dant \\\hline
\end{verbchart}

\begin{verbchart}{coni\=ur\=o, coni\=ur\=are, coni\=ur\=avi, coni\=ur\=atum}{present}{passive}{indicative}
  1 & coni\=uror     & coni\=ur\=amur \\\hline
  2 & coni\=ur\=aris & coni\=ur\=amin\=i \\\hline
  3 & coni\=uratur   & coni\=urantur \\\hline
\end{verbchart}

\begin{verbchart}{am\=o, am\=are, am\=avi, am\=atum}{present}{active}{subjunctive}
  1 & amem    & am\=emus \\\hline
  2 & am\=es  & am\=etis \\\hline
  3 & am\=et  & ament \\\hline
\end{verbchart}

\begin{verbchart}{laud\=o, laud\=are, laud\=avi, laud\=atum}{present}{passive}{subjunctive}
  1 & lauder     & laud\=emur \\\hline
  2 & laud\=eris & laud\=emin\=i \\\hline
  3 & laud\=etur & laudentur \\\hline
\end{verbchart}

\subsection{The Second Conjugation (Long-\=e verbs)}
Second conjugations verbs are characterized by the theme vowel, \=e.  Easily
recognizable second conjugation verbs include \textit{vide\=o, vid\=ere,
v\=id\=i, v\=isum}, and \textit{habe\=o, hab\=ere, habu\=i, habitum}.

\begin{verbchart}{vide\=o, vid\=ere, v\={i}d\=i, v\=isum}{present}{active}{indicative}
  1 & vide\=o   & vid\=emus \\\hline
  2 & vid\=es   & vid\=etis \\\hline
  3 & videt     & vident \\\hline
\end{verbchart}

\begin{verbchart}{fove\=o, fov\=ere, fov\=i, f\=otum}{present}{passive}{indicative}
  1 & foveor    & fov\=emur \\\hline
  2 & fov\=eris & fov\=emin\=i \\\hline
  3 & fovetur   & foventur \\\hline
\end{verbchart}

Unlike the first and third conjugations, and like the fourth conjugation,
the subjunctive for second conjugation verbs is formed with a long \=a 
\textit{after} a shortened theme vowel, in this case short-e.

\begin{verbchart}{habe\=o, hab\=ere, habu\=i, habitum}{present}{active}{subjunctive}
  1 & habeam    & habe\=amus \\\hline
  2 & habe\=as  & habe\=atis \\\hline
  3 & habe\=at  & habeant \\\hline
\end{verbchart}

\begin{verbchart}{fove\=o, fov\=ere, fov\=i, f\=otum}{present}{passive}{subjunctive}
  1 & fovear      & fove\=amur \\\hline
  2 & fove\=aris  & fove\=amin\=i \\\hline
  3 & fove\=atur  & foveantur \\\hline
\end{verbchart}

\subsection{The Third Conjugation (Short-e verbs)}
The third conjugation of Latin verbs is arguably the most regular, and is
characterized by the theme vowel short-e.

\begin{verbchart}{d\=ic\=o, d\=icere, d\=ix\=i, dictum}{present}{active}{indicative}
  1 & d\=ic\=o    & d\=icimus \\\hline
  2 & d\=icis     & d\=icitis \\\hline
  3 & d\=icit     & d\=icunt \\\hline
\end{verbchart}

\begin{verbchart}{ag\=o, agere, \=eg\=i, \=actum}{present}{passive}{indicative}
  1 & agor     & agemur \\\hline
  2 & ageris   & agemin\=i \\\hline
  3 & agetur   & agentur \\\hline
\end{verbchart}

\begin{verbchart}{p\=on\=o, p\=onere, posu\=i, positum}{present}{active}{subjunctive}
  1 & p\=onam    & p\=on\=amus \\\hline
  2 & p\=on\=as  & p\=on\=atis \\\hline
  3 & p\=on\=at  & p\=onant \\\hline
\end{verbchart}

\begin{verbchart}{ed\=o, edere, \=ed\=i, esum}{present}{passive}{subjunctive}
  1 & edar     & ed\=amur \\\hline
  2 & ed\=aris & ed\=amin\=i \\\hline
  3 & ed\=atur & edantur \\\hline
\end{verbchart}

\subsubsection{The Third\textit{-io}}
A subset of the third conjugation, the third-io can
be thought of as part third- and part fourth-conjugation.
It retains the theme vowel short-e but places an i in
some forms where it is not expected for a third conjugation
verb.

\subsection{The Fourth Conjugation (Long-\=i verbs)}
The fourth conjugation is characterized by the theme vowel
long-\=i.

\begin{verbchart}{audi\=o, aud\=ire, aud\=iv\=i, auditum}{present}{active}{indicative}
  1 & audi\=o   & aud\=imus \\\hline
  2 & aud\=is   & aud\=itis \\\hline
  3 & aud\=it   & audiunt \\\hline
\end{verbchart}

\begin{verbchart}{veni\=o, ven\=ire, v\=en\=i, ventum}{present}{active}{subjunctive}
  1 & veniam    & veni\=amus \\\hline
  2 & veni\=as  & veni\=atis \\\hline
  3 & veni\=at  & veniant \\\hline
\end{verbchart}

\begin{verbchart}{senti\=o, sent\=ire, s\=ens\=i, s\=ensum}{present}{passive}{indicative}
  1 & sentior     & sent\=imur \\\hline
  2 & sent\=iris & sent\=imin\=i \\\hline
  3 & sent\=itur   & sentiuntur \\\hline
\end{verbchart}

\begin{verbchart}{senti\=o, sent\=ire, s\=ens\=i, s\=ensum}{present}{passive}{subjunctive}
  1 & sentiar       & senti\=amur \\\hline
  2 & senti\=aris   & senti\=amin\=i \\\hline
  3 & senti\=atur   & sentiantur \\\hline
\end{verbchart}
