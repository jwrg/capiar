\documentclass{reference}
\usepackage{ceystroke}

\begin{document}
\pagenumbering{gobble}		% turn off page numbering
\titletable{Linux Cheat Sheet}{Git, SSH, and Python}
\vspace{1em}

\renewcommand{\arraystretch}{1.6}
\section{Bash (The Bourne Again SHell)}
\subsection{Getting information:}
\refentry{\$ ls }{List the contents of the working directory}
\refentry{\$ tree }{List a tree of files and directories under the working directory}
\refentry{\$ pwd }{Print working directory (i.e., what directory am I in?)}
\refentry{\$ w}{Get information about users currently logged in}
\refentry{\$ who}{Get information about users currently logged in}
\refentry{\$ crontab -l }{List the current cron table}
\refentry{\$ top }{Get realtime information on running processes (press Q to exit)}
\refentry{\$ man <program> }{Open the manual page for \texttt{<program>}}
\subsection{Manipulating files:}
\refentry{\$ mkdir <path> }{Create a new directory at \texttt{<path>} }
\refentry{\$ cp <path> <dest> }{Copy file at \texttt{<path>} to destination \texttt{<dest>}}
\refentry{\$ mv <path> <dest> }{Move file at \texttt{<path>} to destination \texttt{<dest>} (this also works for renaming files)}
\refentry{\$ rm <path>  }{Delete file at \texttt{<path>} \textit{(be careful!)}}
\refentry{\$ nano <path> }{Open the file at \texttt{<path>} using the nano text editor}
\subsection{Changing the system:}
\refentry{\$ crontab -e }{Edit the current cron table with the system default text editor}
\subsection{Close program, or the shell}
\refentry{\keystroke{Ctrl} $+$ \keystroke{c}}{Kill the process running in the shell}
\refentry{\keystroke{Ctrl} $+$ \keystroke{d}}{Quit the current shell session}
\refentry{\$ exit}{Quit the current shell session}

\section{Git}
\subsection{General commands:}
\refentry{\$ git status}{Check status of the git repository in the working directory}
\refentry{\$ git log}{View all the commits for the current repsitory (this may open a pager if there are numerous commits)}
\refentry{\$ git diff}{View all the unstaged changes made since the last commit}
\refentry{\$ git remote -v show}{View information on remote sources for the repository}
\refentry{\$ git remote add <alias> <URL> }{Add a remote location for the repository at \texttt{<URL>} under the alias \texttt{<alias>}}
\subsection{When making changes to code:}
\refentry{\$ git add <path>}{Add the path \texttt{<path>} to the staging area to be included in the next commit}
\refentry{\$ git add .}{Add everything in the current directory to the staging area to be included in the next commit}
\refentry{\$ git commit -m <string>}{Commit all staged changes and include a message, \texttt{<string>}}
\refentry{\$ git commit -v}{Commit all staged changes and open the system default text editor to author a commit message}
\refentry{\$ git push origin master}{Upload the most recent commit(s) on the master branch to the origin}
\subsection{When downloading the code from the cloud:}
\refentry{\$ git clone <URL>}{Download the repository located at \texttt{<URL>} to the working directory, in a directory with the repository's name}
\refentry{\$ git pull}{Download the most up-to-date version of the current repository via the origin}

\section{SSH}
\refentry{\$ ssh <host>}{Open a remote shell on \texttt{<host>} under the default user for \texttt{<host>}}
\refentry{\$ ssh <user>@<host>}{Open a remote shell on \texttt{<host>} under the user \texttt{<user>}}

\section{Python}
\refentry{\$ python }{Open the Python interpreter}
\refentry{\$ python <path> }{Run the file at \texttt{<path>} using Python}

\end{document}
